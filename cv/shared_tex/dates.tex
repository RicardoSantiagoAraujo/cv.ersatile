%=================== AGE
\AfterPreamble{
  %Age calculation:
  \newcounter{dateone}%
  \newcounter{datetwo}%
  \setmydatenumber{dateone}{\birthYear}{\birthMonth}{\birthDay}%
  \setmydatenumber{datetwo}{\the\year}{\the\month}{\the\day}%
  \FPsub\result{\thedatetwo}{\thedateone}
  \FPdiv\myAge{\result}{365.2425}
  \newcommand{\myAgeFloor}{\FPtrunc\myAge{\myAge}{0}\myAge}% round age
}




%=================== DATES
\usepackage{datetime2}
\usepackage{datetime2-calc}

\DTMnewdatestyle{excludeDay}{%
  \renewcommand{\DTMdisplaydate}[4]{%
    % This style excludes the day
    \DTMmonthname{##2}\nobreakspace%  (full) Month
    % \number##3,\space%                day,
    \number##1%                       year
  }%
  \renewcommand{\DTMDisplaydate}{\DTMdisplaydate}%
}

\DTMnewdatestyle{includeDay}{%
  \renewcommand{\DTMdisplaydate}[4]{%
    % This style includes the day
    \DTMmonthname{##2}\nobreakspace%  (full) Month
    \number##3,\space%                day,
    \number##1%                       year
  }%
  \renewcommand{\DTMDisplaydate}{\DTMdisplaydate}%
}


% Set date style for whole doc
\DTMsetdatestyle{excludeDay}


% Command to format the date after splitting it
% and apply default format (-1) if no format is specified
\newcommand{\customDateFormat}[2][excludeDay]{%
    \StrBefore[1]{#2}{-}[\MyYear]% Extract year
    \StrBetween[1, 2]{#2}{-}{-}[\MyMonth]% Extract month
    \StrBehind[2]{#2}{-}[\MyDay]% Extract day
    %
    % Print for testing
    % Year: \MyYear \newline
    % Month: \MyMonth \newline
    % Day: \MyDay \newline
    %
    \DTMsetdatestyle{#1}% Specify format
    \DTMdisplaydate{\MyYear}{\MyMonth}{\MyDay}\xspace% Display the date with the specified format
}

\usepackage{intcalc}
% Calculate date difference
\newcommand{\dateDifference}[2]{%
  % Saving the two dates
  \DTMsavedate{mydate1}{#1}%
  \ifthenelse{%
    \equal{#2}{}}% if second value is empty
    {% IF
      \DTMsavenow{mydate2}%
    }%
    {% ELSE
      \DTMsavedate{mydate2}{#2}%
    }%
  % Calculating the difference between the two dates
  \newcount\dayCount%
  \DTMsaveddatediff{mydate2}{mydate1}{\dayCount}%
  %
  % Display difference
  %%%% IN DAYS
  % \pgfmathsetmacro{\dayCount}{\dayCount}% count from 1st day of month
  % \number\dayCount\ days%
  %%%% IN MONTHS (APPROXIMATION)
  % \pgfmathsetmacro{\monthCount}{\dayCount/30.44 + 1}% +1 to round up
  % \FPtrunc\monthCount{\monthCount}{0}\monthCount\ months%
  %%%% IN YEARS (APPROXIMATION)
  % \pgfmathsetmacro{\yearCount}{\dayCount/365.25 + 0.1}% +0.1 to round up
  % \FPtrunc\yearCount{\yearCount}{1}\yearCount\ years%
  %%%% IN YEARS, MONTHS (APPROXIMATION)
  \newcommand{\yearPart}{\intcalcDiv{\dayCount+30}{365}}% add + 30 to dayCount to include current month
  \newcommand{\yearModule}{\intcalcMod{\dayCount+30}{365}}% add + 30 to dayCount to include current month
  \newcommand{\monthPart}{\intcalcDiv{\yearModule}{30}}%
  %%%% Open parenthesis if at least 1 month has passed
  \ifnum\yearPart>0 $($\else\ifnum\monthPart>1$($\fi\fi%
  % ------------ years
  \ifnum\yearPart>0
    \yearPart
    \ifnum\yearPart>1
      \ \myYearPlural%
    \else
      \ \myYearSingular%
    \fi%
  \fi
  % separator
  \ifnum\yearPart>0
    \ifnum\monthPart>0
      , %
    \fi%
  \fi
  % ------------ months
  \ifthenelse{\yearPart>0}
  {% above 1 year
    \ifnum\monthPart>0
      \monthPart
      \ifnum\monthPart>1
        \ \myMonthPlural%
      \else
        \ \myMonthSingular%
      \fi%
    \fi%
  }
  {% below 1 year
    % do not print if it only one month
    \ifnum\monthPart>1
      \monthPart\ \myMonthPlural%
    \else%
      % DO NOT PRINT%
    \fi%
  }%
  %%%% Close parenthesis if at least 1 month has passed
  \ifnum\yearPart>0 $)$\else\ifnum\monthPart>1$)$\fi\fi%
}

% Center align smaller text in a line
\newcommand*\raiseup[2]{%
        \begingroup
        \setbox0\hbox{#1\strut #2}%
        \leavevmode
        \raise\dimexpr \ht\strutbox - \ht0\box0
        \endgroup
}

% Display Date range
% Define a new command \dateRange that takes multiple arguments
\NewDocumentCommand{\dateRange}{O{excludeDay} O{isIncludeDateDiff} m m}{
    % Optional: #1 date format (otherwise uses excludeDay, which is the chosen default)
    % Optional: #2 whether to display date range or not (default is set as a variable elsewhere)
    % Mandatory: #3 beginning date
    % Mandatory: #4 end date
    \customDateFormat[#1]{#3} %
    \ -- %
    \ifthenelse{
        \equal{#4}{}}% if second value is empty
        {% IF
          \myPresent%
        }%
        {% ELSE
          \customDateFormat[#1]{#4}%
        }%
        % Show date difference
        \ifthenelse{
        \boolean{#2}}{% if second value is empty} 
          \color{colorGrayDark}\ %
          \raiseup{\scriptsize}{%
            % (%
            \dateDifference{#3}{#4}%
            % )%
          }%
        }
        {}
        }
