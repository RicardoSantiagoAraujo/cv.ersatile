
% Vertical separation between tables DEFAULT /!\ May be overwritten downstream
\newcommand{\myTablesSeparator}{\vspace{30pt}}


\NewDocumentEnvironment{myTableEnv}
{
  O{\myTableWidth}% table width
  m % column definitions
}
% \newenvironment{myTableEnv}
{%
  % \table%
  \noindent\tabularx{#1}{#2}%
  \hline% first horizontal line
}
{%
  \endtabularx%
  % \endtable%
  \par%
}

%%% Table lines (horizontal/vertical):
\setlength\arrayrulewidth{0.1pt}% thickness of rule
\arrayrulecolor{colorTableBorders}% color of rule
% % Change space above/below
% \let\savehline\hline%
% \def\hline{%
%   % by subtracting arrayrulewidth, the ruler adds no height
%   \noalign{\vskip-0.5\arrayrulewidth}% space above
%   \savehline%
%   \noalign{\vskip-0.5\arrayrulewidth}% space below
% }%

\NewDocumentCommand{\myRow}
{%
  m% 1 left label
  m% 2 left content
  m% 3 right label
  m% 4 right content
}
{%
    % \hline%
    %%% LEFT ===================
    \myLabelValueCellPair{#1}{#2}
    &% .................
    %%% RIGHT ===================
    \myLabelValueCellPair{#3}{#4}
    \\%
    \hline%
}

\NewDocumentCommand{\myLabelValueCellPair}
{%
  m% 1 label
  m% 2 content
}
{%
  {%
    % \hfill %  push right /!\ DO NOT do alignment here. Better to do it in the table definitions for flexibility as this cannot be overwritten
    \setColorCell%
    % \scshape%
    #1% Right Label
  }%
  &% ..................
  {%
    \setColorCell%
    % \bfseries%
    #2% Right Content
  }%
}



\NewExpandableDocumentCommand{\myCell}
{
  m% 1 number of columns to use up
  O{c}% alignment (c)enter, (l)eft, (r)ight, p(normal), m(iddle), b(ottom), etc
  m% content
}
{%
  \ifnum#1>1%
    % if multiple columns are required
    \multicolumn%
    {#1}%
    {>{\centering\hsize=\dimexpr#1\hsize+#1\tabcolsep+\arrayrulewidth\relax}#2}%
    {%
      \setColorCell%
      #3%
    }%
  \else%
    % if only one column is required
    \setColorCell%
    #3%
  \fi%
}

%%% command to sum values in a list eg {1,2,3} returns 6
\newcommand\sumlist[1]{\the\numexpr\sumlistaux#1,\relax\relax}
\def\sumlistaux#1,#2\relax{#1\ifx\relax#2\else+\sumlistaux#2\relax\fi}


\newcommand{\myTableContentsSpaceAboveAdj}{-5pt} % Table content vertical separator DEFAULT /!\ POSSIBLY OVERWITTEN DOWNSTREAM
%%% BODY OF SECTION CONTENTS USED INSIDE A TABLE TO FILL WHOLE WIDTH
\newcommand{\myTableContents}[1]{%
    \multicolumn{4}%%% number should correspond to the default total number of columns
    {@{}p{\myRowWidth}@{}}%
    {%
        \vspace{\myTableContentsSpaceAboveAdj}%
        \myJustification%
        #1%
    }%
}


% Line separator DEFAULT /!\ POSSIBLY OVERWITTEN DOWNSTREAM\
\newcommand{\myhlinePadding}{2mm}%
\newcommand{\myhline}{
  \noalign{\vskip+\myhlinePadding}% space above
  \hline%\hline%
  \noalign{\vskip+\myhlinePadding}% space below
}

%%% Specific macro for misc section rows
\NewDocumentCommand{\miscSkillRow}
{
  m% label
  m% content
}
{%
  {%
  % \bfseries%
  #1}%
  &%
  #2\\%
}