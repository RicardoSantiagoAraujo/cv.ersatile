
% ENGLISH VERSION (DEFAULT)
\newcommand{\myCuriosity}{Curiosity}
\newcommand{\myRigor}{Rigor}
\newcommand{\myAdaptability}{Adaptability}
\newcommand{\myPedagogy}{Pedagogy}
\newcommand{\myOrganisation}{Organisation}
\newcommand{\myCollaboration}{Collaboration}

\ifthenelse{\equal{\myLanguage}{fr}}{
    % FRENCH VERSION
    \renewcommand{\myCuriosity}{Curiosité}
    \renewcommand{\myRigor}{Rigueur}
    \renewcommand{\myAdaptability}{Adaptabilité}
    \renewcommand{\myPedagogy}{Pédagogie}
    \renewcommand{\myOrganisation}{Organisation}
    \renewcommand{\myCollaboration}{Collaboration}
}{}




% Define key/value pairs to include in the diagram
% Value is between 0 and 1
\newcommand{\myValues}{
    \myAdaptability/1,
    \myOrganisation/0.7,
    \myRigor/1,
    \myPedagogy/0.9,
    \myCollaboration/0.8,
    \myCuriosity/1
    }

% define size of circle
\newcommand{\maxRadius}{1}
% choose how many cicles to divide it into
\newcommand{\numberOfCircles}{5}


% Initialize a counter to store the length of the list
\newcounter{listlength}

% Define a macro to calculate the length of a list
\newcommand{\getListLength}[2]{%
    \setcounter{#1}{0}% Reset the counter
    \foreach \item in #2 {%
        \stepcounter{#1}% Increment the counter for each element
    }%
}

% Use the macro to calculate the length of \myList
\getListLength{listlength}{\myValues}

% Output the length of the list
% The length of the list is: \thelistlength.




\begin{tikzpicture}[scale=1] % Adjust the scale factor as needed
    %%%%% 0,0 is center

    % DRAW CIRCLES
    \pgfmathsetmacro{\step}{\maxRadius / \numberOfCircles}
    \foreach \r in {0,\step,...,\maxRadius}{
        \draw[opacity=0.15] (0,0) circle (\r cm);}



    % FILL WITH INFO
    \foreach \key/\val [count=\i from 0, count=\j] in \myValues
    {
        \pgfmathsetmacro{\angle}{360 / \thelistlength}
        \pgfmathsetmacro{\k}{int(Mod(\i+1,\thelistlength))}

        % ADD CROSSLINE
        \draw[opacity=0.5] (0,0) -- ++ ({\k*\angle}:\maxRadius) coordinate (lab);

        % ADD LABELS
        \path[opacity=1] (0,0) -- ++ (\k*\angle:\maxRadius) coordinate[label=\k*\angle: \scriptsize\key] ();

        % ADD DOTS
        \node (A\i) [
            circle,
            draw,
            semithick,
            fill=white,
            inner sep=2pt,
            opacity=0.6,
            anchor=center
            ] at (\k*\angle:\maxRadius*\val) {};

        % ADD NODES
        \node (A\i) [anchor=center, inner sep=0pt] at (\k*\angle:\maxRadius*\val) {};
    }


    \pgfmathsetmacro{\upperlimit}{\thelistlength - 1}
    \foreach \i in {0,...,\upperlimit}
    {
        \pgfmathsetmacro{\k}{int(Mod(\i+1,\thelistlength))}

        % OUTER CONNECTING LINE
        \draw[
            very thick,
            blue,
            shorten <=2pt,
            shorten >=2pt
            ] (A\i) -- (A\k);



        % INNER FILL
        \draw[draw = none, fill=gray, opacity=0.3] (A\i) -- (0,0)  -- (A\k);

        % ADD CIRCLES
        \node (DOT\i) [
            circle,
            inner sep=0.1pt,
            draw,
            minimum size=0.2cm,
            semithick,
            fill=white
            ] (myNode) at (A\i) {};
    }
\end{tikzpicture}
